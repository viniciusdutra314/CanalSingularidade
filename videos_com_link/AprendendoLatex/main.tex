\documentclass[a4paper]{article}
\usepackage[utf8]{inputenc}
\usepackage{algorithmic}
\usepackage{float}
\usepackage{graphicx}
\usepackage{tcolorbox}
\usepackage{amsmath}
\usepackage{physics}
\usepackage{commath}
\usepackage{booktabs}
\usepackage{amssymb}
\usepackage{caption}
\usepackage{subcaption}
\usepackage[bottom=2.0cm,top=2.0cm,left=2.0cm,right=2.0cm]{geometry}
\usepackage[portuges]{babel}
\usepackage{indentfirst}
\usepackage{hyperref}  %%%%
\usepackage{comment}
\hypersetup{colorlinks,citecolor=black,filecolor=black,linkcolor=black,urlcolor=black} %%%%


\begin{document}
\title{Relatório de Laboratório de Física II}

\begin{titlepage}
	\begin{center}
		\begin{figure}[htb!]
		\centering
				\includegraphics[scale=1.5]{imagens/logo.jpg}
		\end{figure}
		\vspace{20pt}
        \Large{\textbf{Nome Universidade }}\\
        \Large{Nome Instituto}\\
        \Large{Nome da matéria}\\
        
        \vspace{150pt}
        
        \LARGE{\textbf{Prática X}}\\ 
        \Large{Nome da prática}\\ %Entre com o título do experimento
        
        \vspace{125pt}
        
        \hfill Grupo composto por: \\
        
        \vspace{20pt} 
        \hfill Estudante 1 \hspace{20pt}Nº Matrícula\\
        \hfill Estudante 2 \hspace{20pt}Nº Matrícula\\
        \hfill Estudante 3 \hspace{20pt}Nº Matrícula\\

        \vspace{25pt}
        \hfill {Professor:}\\
        
        
        \vspace{\fill}
        \Large \bf{Cidade\\}
        \Large \bf{Data}
          
	\end{center}
\end{titlepage}



\newpage

\tableofcontents
\thispagestyle{empty}

\newpage


\section{Templates}
Existem muitos templates que você pode usar como base para o seu arquivo, entre nesse \href{https://www.overleaf.com/latex/templates}{link}
\section{Pacotes}
Logo apos o documentclass é interessante importar todas as bibliotecas/pacotes, é só colocar barra usepackage{nome do pacote}. É muito parecido com import em linguagens de programação
\section{Como fazer Tópicos}
É possível criar secções usando o barra section{}. Também é possível escrever em \textbf{Negrito}, o LaTex também tem um corretor de texto automático. Talvez seja interessante minimizar algumas janelas para ter mais espaço para a edição. 

Clicando duas vezes em uma parte do texto, vamos diretamente na parte equivalente no código. O botão Recompile pode ser atualizado automaticamente
\subsection{Muda toda numeração se apagar}
\subsection{Subtópico}
O subtópico funciona da mesma forma com tópico + ponto + subtópico
\subsubsection{Subsubtópico}
\section{Equações}
É possível criar equações começando com begin equation e terminando end equation, você pode treinar como escrever equações nesse \href{https://editor.codecogs.com/?lang=pt-pt}{link}
\begin{equation}
    E=mc^2
\end{equation}
Também é possível escrever equações junto ao texto $F=ma$ começando e encerrando com cifrão. Às vezes você quer fazer uma demonstração sem numerar cada equação, para isso use * depois de begin e end equation (De forma é possível tirar a numeração de qualquer coisa). Aqui um exemplo
\begin{equation}
    \label{eqn:comp_onda1}
    n \frac{\lambda_n}{2} = L \implies \lambda_n = \frac{1}{n} 2L
\end{equation}

\begin{equation*}
    \nu = \lambda_n f_n \implies \nu = \frac{1}{n} 2L f_n
\end{equation*}

\begin{equation*}
    n \frac{\nu}{2L} = f_n
\end{equation*}

\begin{equation}
    \label{eqn:modo_fundamental}
    f_1 = \frac{\nu}{2L}
\end{equation}

\begin{equation}
    \label{eqn:f_n}
    f_n =  n f_1
\end{equation}
\section{Teoremas}
Precisamos do pacote amsthm, com ele conseguimos criar teoremas, corolários, lemas enfim. Precisamos definir o que vamos usar com barra newtheorem como segue no código

\newtheorem{theorem}{Teorema}
\newtheorem{corollary}{Corolário}
\newtheorem{Lemma}{Lema}
\begin{theorem}[Lei dos Cossenos]
$$a^2=b^2+c^2 -2bc*cos(\alpha)$$
\end{theorem}


\begin{corollary}
Quando $\alpha=\pi/2$, temos que $a^2=b^2+c^2$
\end{corollary}

\begin{Lemma}
Esse aqui é um exemplo de lema
\end{Lemma}

\begin{theorem}
    Outro teorema ai 
 $\int_{a}^{b}F'(x)dx=F(b)-F(a)$
\end{theorem}
\section{Fazer uma citação}
Esse é um exemplo de citação, é só usar o ambiente quote
\begin{quote}
    “Sometimes it happens that a man's circle of horizon becomes smaller and smaller, and as the radius approaches zero it concentrates on one point. And then that becomes his point of view.” -- David Hilbert
\end{quote}
\section{Figuras}
Para adicionar imagens/figuras o procedimento é análogo ao de equações, begin figure, end figure. em includegraphis coloque o caminho para a imagem, [scale] é possível mudar o tamanho da imagem, angle a inclinação da imagem. Em caption coloque uma legenda para a figura, caso queira ocultar a numeração coloque * antes do caption. As imagens podem aparecer em lugares \textbf{indesejados}, para resolver isso coloque [H] depois de figure
\begin{figure}[H]
    \centering \includegraphics[scale=0.25,angle=10]{imagens/thumbnail.jpg}
    \caption{Legenda para a sua imagem}
    \label{thumbnail}
\end{figure}

Você pode criar gráficos no matplotlib, geogebra e importar eles como figuras.

\begin{figure}[H]
    \centering \includegraphics[scale=0.5]{imagens/gráfico.png}
    \caption{Oscilação na Água}
    \label{thumbnail}
\end{figure}


\subsection{Citar equações ou figuras}
É possível fazer referencia a uma figura, por exemplo "Como mostra a figura \ref{thumbnail}" ou "Como mostra a equação \ref{eqn:f_n}". É só você criar labels e usar o ref para referenciar
\section{Tabelas}
 A explicação está no próprio código da tabela
\begin{table}[H]
\centering
\begin{tabular}{l|c r}
l=left & c=center & r=right \\\hline
$|$ $==>$ & 30 & $\int xdx$\\
-3 & $\alpha$ & 15 \\
Vamos & acrescentar & hline \\\hline
É possível & continuar & a tabela
\end{tabular}
\label{tabela}
\caption{Coloque um titulo para a tabela}
\end{table}
\subsection{Outros exemplos}
\begin{table}[h]
\centering
\caption{Example of a Complex Table}
\label{tab:complex_table}
\begin{tabular}{|c|c|c|c|}
\hline
\textbf{Name} & \textbf{Subject} & \textbf{Score} & \textbf{Grade} \\ \hline
John Doe & Mathematics & 87 & B+ \\ \hline
Jane Doe & Physics & 92 & A- \\ \hline
Jim Smith & Chemistry & 83 & B \\ \hline
Emily Davis & Biology & 76 & C+ \\ \hline
\end{tabular}
\end{table}


\begin{table}[h]
\centering
\caption{Example of a Table with 8 Columns}
\label{tab:table_with_8_columns}
\begin{tabular}{|c|c|c|c|c|c|c|c|}
\hline
\textbf{Column 1} & \textbf{Column 2} & \textbf{Column 3} & \textbf{Column 4} & \textbf{Column 5} & \textbf{Column 6} & \textbf{Column 7} & \textbf{Column 8} \\ \hline
1 & 2 & 3 & 4 & 5 & 6 & 7 & 8 \\ \hline
9 & 10 & 11 & 12 & 13 & 14 & 15 & 16 \\ \hline
17 & 18 & 19 & 20 & 21 & 22 & 23 & 24 \\ \hline
\end{tabular}
\end{table}
\section{Listas}
Para criar uma lista basta seguir esses passos:
\begin{enumerate}
    \item begin enumerate e feche com enumerate
    \item escreva barra item para adicionar um novo item
    \item a numeração é automática assim como tudo em latex
    \item olhe o código para ver o que eu estou falando
\end{enumerate}
\section{Referências Bibliográficas}
Texto hipotético "Como já foi demonstrado experimentalmente
\cite{chartas_measuring_2017}
os buracos negros tem propriedades estranhas \cite{lasota_extracting_2014}". 

É necessário criar um arquivo .bib, as referencias podem ser extraídas de programas como zotero/mendeley ou do próprio google acadêmico. Para citar é só usar barra cite(autor e ano)

\section{Cor}
Precisamos importar o pacote xcolor, no começo do documento temos usepackage{xcolor}
e com ele podemos escrever
\textcolor{red}{co}\textcolor{green}{lo}\textcolor{blue}{rido}
\section{Hyperlinks}
Hyperlinks são textos clicáveis, é possível acessá-los também em PDF, é necessário ter iniciado no começo do código usepackage{hyperref}. Eu quero deixar algumas recomendações, aqui está a série (em inglês) de Latex feito pelo 
\textbf{
\href{https://www.youtube.com/watch?v=xyZtxfMsD38&list=PLHXZ9OQGMqxcWWkx2DMnQmj5os2X5ZR73&index=8}{Dr Trefor}} e em português temos os vídeos da \textbf{\href{https://www.youtube.com/watch?v=zR-QuNf3agQ&list=PLb735fZHArLamJiCIXsQT6BiHM1IgYQ43}{Jaqueline}}

Também recomendo o ChatGPT, querendo ou não Latex é muito próximo de uma linguagem de programação e o ChatGPT é incrível para explicar códigos.


\bibliographystyle{plain}
\bibliography{referencias.bib}
\vspace{50pt}

\end{document}
